\section{Giới thiệu về đề tài}
\label{sec:introduction}

\subsection{Đặt vấn đề}
Với sự bùng nổ của dữ liệu điện tử, đặc biệt là dữ liệu dạng văn bản trong thời
đại internet, chúng đem lại nhiều cơ hội và thách thức cho việc nghiên cứu và
phân tích dữ liệu. Liệu con người có bị chìm trong lượng thông tin khổng lồ như
vậy không? Liệu chúng ta có cách nào đó sử dụng sức mạnh tính toán của máy
tính, tự động hoá công việc phân tích các dữ liệu để đưa ra các thông tin cần
thiết và hữu ích?

Bên cạnh đó, sự phát triển của mạng xã hội trong thập kỷ 21 đã dần hình thành
nên các hành vi tâm lý của người dùng. Đặc biệt trong số đó là hội chứng sợ bỏ
lỡ (fear of missing out, hay FOMO), là cảm giác bứt rứt hoặc ám ảnh rằng bản
thân đang bỏ lỡ những việc mà những người cùng trang lứa có thể làm được hoặc
làm tốt hơn \cite{jwtintelligence_fear_2015}. Chính vì các hành vi tâm lý này,
các mạng xã hội, cụ thể là các nhà khoa học dữ liệu đã tận dụng dữ liệu để giúp
người dùng có thể tìm kiếm thông tin một cách nhanh chóng và hiệu quả. Đơn cử
là mạng xã hội Twitter với tính năng phát hiện Xu hướng (Trending now), hay
Đang xảy ra (What's Happening) giúp họ trở thành mạng xã hội phổ biến để cập
nhật tin tức từ chính trị, xã hội đến đời sống cá nhân của những người nổi
tiếng.

Để phát hiện sự thay đổi của các đề tài đang nổi bật trong đại chúng, về cơ bản
ta phải phát hiện được chủ đề của các văn bản, tức là phân tích xem họ đang
viết về nội dung gì.  Tiếp theo ta cần phân tích mức độ phổ biến của các chủ đề
đó, hay phạm vi mà chủ đề đó đang được phát triển để quyết định xem các chủ đề
đó có đáng được quan tâm hay không.

\# TODO viet tiep

\subsection{Mục tiêu}
\begin{itemize}
	\item Tìm hiểu các cách thức phát hiện xu hướng dựa trên các mô hình xác
		suất hoặc mô hình hướng dữ liệu, đơn cử là mô hình phân phối Poisson.

	\item Tìm hiểu các mô hình phát hiện chủ đề (Latent Dirichlet Allocation
		hay LDA,...).

	\item Tìm hiểu các mô hình phát hiện chủ đề dựa trên ngữ nghĩa (BERT,
		PhoBERT,...)
\end{itemize}

\section{Các mô hình}
\label{sec:models}

\subsection{Phát hiện xu hướng}
Mô hình phát hiện xu hướng được....

\subsection{Phát hiện chủ đề}

\subsection{Phân tích ngữ nghĩa}

\subsection{VnCoreNLP}

\section{Thử nghiệm}
\label{sec:experiments}
