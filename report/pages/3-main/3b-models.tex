\chapter{Các mô hình}
\label{sec:models}

Ta thường có hai cách để phát hiện các sự kiện hay chủ đề đang nổi bật:
\textbf{(1)} hồi tưởng (retrospect), sử dụng toàn bộ dữ liệu để làm đầu vào cho
mô hình, hoặc, \textbf{(2)} online, phát hiện trên thời gian thực bằng cách đưa
vào (feed) mô hình mỗi khi ta thu thập được dữ liệu
mới~\cite{lauOnlineTrendAnalysis2012}. Với cách \textbf{(1)}, ta có thể phân
tích được các chủ đề hay sự kiện trong lịch sử, và thường được hiện thực bằng
các phương thức như gọm cụm (clustering) hay phát hiện bất thường (anomaly
detection). Tuy nhiên, nếu ta muốn phát hiện các sự kiện đang xảy ra trong thời
gian thực, chẳng hạn như phát hiện động đất, thì ta cần dùng đến kỹ thuật thứ
\textbf{(2)}.

Ba mô hình chính được báo cáo này nhắc tới bao gồm mô hình phát hiện xu hướng
(được xếp vào kỹ thuật thứ \textbf{2}), mô hình phát hiện chủ đề và mô hình
phát hiện chủ đề dựa trên ngữ nghĩa (được xếp vào kỹ thuật thứ \textbf{1}).  Mô
hình phát hiện xu hướng được nghiên cứu là mô hình được phát triển bởi
Twitter~\cite{hendricksonTrendDetectionSocial2015}, bao gồm mô hình xác suất và
mô hình thiên dữ liệu (data-driven). Mô hình phát hiện chủ đề là mô hình khai
thác sự xuất hiện của các từ khoá trong các tập văn bản để tìm ra sự tương quan
giữa các từ khoá và sự ảnh hưởng của từ khoá đối với chủ đề của văn bản. Mô
hình phát hiện chủ đề mà báo cáo này sẽ tập trung nghiên cứu là mô hình Latent
Dirichlet Allication (LDA). Mô hình phát hiện chủ đề dựa trên ngữ nghĩa là một
mô hình phát hiện chủ đề nhưng được dựa trên ngữ nghĩa của văn bản, hay ngữ
cảnh mà văn bản đó biểu hiện để thực hiện gom cụm chủ đề. Mô hình ngữ nghĩa
BERT và PhoBERT là đối tượng nghiên cứu và là cơ sở cho thực nghiệm, được kết
hợp với LDA và phương pháp gom cụm để có thể sinh ra kết quả tốt
hơn~\cite{lamGomCumVan2021}.



\section{Mô hình phát hiện xu hướng}
Tương tác của người dùng trên mạng xã hội luôn có sự liên quan nhất định nào đó
đến các sự kiện ở ngoài thế giới
thực~\cite{hendricksonTrendDetectionSocial2015}. Để phát hiện xu hướng, ta cần
phải trả lời các câu hỏi như: sự kiện đó bắt đầu từ khi nào? Mức độ thay đổi
phạm vi của sự kiện này là lớn hay nhỏ? Và chúng thay đổi như thế nào đối với
các sự kiện bình thường (typical) khác? Việc này không chỉ giúp phân biệt giữa
các sự kiện bình thường và các sự kiện bất thường (atypical) mà còn giúp so
sánh các sự kiện bất thường với
nhau~\cite{hendricksonTrendDetectionSocial2015}.

Twitter là mạng xã hội cho phép người dùng đăng các bài đăng, hay còn được gọi
là Tweet. Trong một Tweet, người dùng có thể thêm văn bản (giới hạn 280 ký tự)
hoặc hình ảnh. Mô hình của Twitter định lượng các hành vi người dùng bằng cách
đếm hashtag, mention hoặc liên kết (link) trong một khoảng thời gian cố định
(bucketed count). Khi một định lượng được thể hiện bằng một từ hoặc cụm từ, ta
có thể gọi từ hoặc cụm từ đó là \textit{chủ
đề}~\cite{hendricksonTrendDetectionSocial2015}.

Tuy nhiên, ta không thể biết trước được mức độ của sự thay đổi, hay thời gian
mà sự thay đổi đó sẽ xảy ra. Có sự kiện chỉ diễn ra trong một vài giây, có sự
kiện có thể diễn ra trong hàng năm. Hơn nữa, định lượng cho các sự kiện đó có
thể thay đổi từ một vài Tweet cho đến hàng triệu Tweet. Để phát triển một giải
thuật có thể đáp ứng được các kích thước dữ liệu trên phạm vi rộng lớn là việc
không hề đơn giản~\cite{hendricksonTrendDetectionSocial2015}.

Nhiều kỹ thuật dùng để phát hiện xu hướng đều định nghĩa một mô hình cơ sở
(background model), là mô hình có thể đại diện cho giả thuyết không (null
hypothesis), về cơ bản có nghĩa là \textit{không trend} (không là xu hướng).
Những thay đổi của các định lượng so với mô hình cơ sở (deviation) được tính
toán thành một hệ số (figure-of-merit) $\eta$. Giá trị $\eta$ càng lớn thì sự
thay đổi của định lượng so với mô hình cơ sở càng nhiều. Và một giá trị $\eta$
được định nghĩa trước có thể được sử dụng để chấp nhận hoặc từ chối giả thuyết
không~\cite{hendricksonTrendDetectionSocial2015}.

Một kỹ thuật khác có thể bao gồm thành phần cơ sở (non-trend, hay \textit{không
trend}) và thành phần giống trend (trend-like, hay \textit{đang là xu hướng}).
Khi đó, giá trị $\eta$ sẽ thể hiện cho việc dữ liệu sẽ giống \textit{trend} hơn
hay dữ liệu sẽ giống \textit{không trend}
hơn~\cite{hendricksonTrendDetectionSocial2015}.

\subsection{Mô hình xác xuất Poisson}

Phân phối Poisson là phân phối xác suất rời rạc, dùng để thể hiện xác suất số
lần một sự kiện nào đó xảy ra trong một khoảng thời gian nhất
định~\cite{haightHandbookPoissonDistribution1967}. Vì vậy, phân phối Poisson có
khả năng áp dụng cho việc định lượng dữ liệu trên mạng xã hội, hay nói cách
khác, ta có thể mặc định dữ liệu đó tuân theo phân phối
Poisson~\cite{hendricksonTrendDetectionSocial2015}.

Ví dụ, ta xét số lượng Tweet có chứa hashtag ``\#covid19'' trong một khoảng thời
gian nhất định. Tần suất đăng bài của người dùng có thể thay đổi theo thời
gian. Nhưng nếu ta bỏ qua sự thay đổi đó, ta có thể nói, số lượng Tweet có chứa
hashtag ``\#covid19'' tuân theo phân phối Poisson:
\[ P\left(c_{i} ; \nu\right)=\nu^{c_{i}} \cdot e^{-\nu} / c_{i} ! \]

với:
\begin{itemize}
	\item $c_{i}$ là số lượng Tweet có chứa hashtag ``\#covid19'' trong một khoảng
	thời gian nhất định.
	\item $P$ là xác suất xuất hiện các Tweet có chứa hashtag ``\#covid19'' với số
	lượng $c_{i}$.
	\item $\nu$ là số lượng kỳ vọng các Tweet có chứa hashtag ``\#covid19''.
\end{itemize}

Vì ta không thể biết chính xác giá trị $\nu$, nên ta có thể lấy số lượng Tweet ở
khoảng thời gian trước đó (đã biết) là $c_{i - 1}$ để biểu diễn cho $\nu$. Ta có
hệ số $\eta$ tại một thời điểm nào đó là:
\[ c_{i}=\eta \cdot \mathrm{CI}(\alpha, \nu)+\nu \]

với:
\begin{itemize}
	\item $\nu=c_{i-1}$.
	\item $CI(\alpha, \nu)$ là khoảng tin cậy (confidence interval) cho phân
	phối Poisson với mean $\nu$ và độ tin cậy $\alpha$.
\end{itemize}

Khi đó, một số lượng Tweet $c_{i}$ được định nghĩa để có thể từ chối giả thuyết
khi:
\[
	c_{i}>=\eta_{c} \cdot \operatorname{CI}\left(\alpha, c_{i-1}\right)+c_{i-1}
\]

với $\eta_{c}$ và $\alpha$ là các tham số để điều chỉnh hiệu suất của thuật
toán~\cite{hendricksonTrendDetectionSocial2015}.

Trên thực tế, dữ liệu thường không tuân theo phân phối Poisson với mean chỉ đơn
giản là một điểm dữ liệu. Để giải quyết vấn đề này, ta có thể tính mean bằng
giá trị trung bình của nhiều điểm dữ liệu trong lịch sử trong cùng một sliding
window. Tuy nhiên, ta không thể bỏ qua tần suất hoạt động của người dùng trong
ngày, trong tháng hoặc trong năm. Chẳng hạn như, người dùng có xu hướng đăng
Tweet nhiều nhất trong ngày vào lúc 17h-20h. Cho nên, khi ta bắt gặp sự thay
đổi đáng kể trong khoảng thời gian đó, ta không thể đơn giản kết luận đó là xu
hướng.

Mô hình trên khá đơn giản và chỉ yêu cầu một điểm hoặc vài điểm dữ liệu cho mô
hình cơ sở. Tuy nhiên với lượng dữ liệu thay đổi liên tục và thay đổi trong
phạm vi rộng, ta không thể tìm được một cặp tham số $\eta$ và $\alpha$ tối ưu
cho mọi loại dữ liệu.

\subsection{Mô hình thiên dữ liệu (data-driven)}
Để giải quyết các vấn đề của mô hình Poisson,
\cite{hendricksonTrendDetectionSocial2015}~đề xuất phương pháp đơn giản hơn đó
là so sánh dữ liệu cần phân tích với các dữ liệu đã được gán nhãn. Trước tiên,
ta phải phân loại dữ liệu có sẵn thành \textit{đang trend} hoặc \textit{không
trend}. Sau đó, ta định nghĩa một hàm tính khoảng cách từ dữ liệu đã gán nhãn
và dữ liệu ta cần phân tích:
\[ d(r, s)=\sum_{i}^{N}\left(r_{i}-s_{i}\right)^{2} \]

với:
\begin{itemize}
	\item $r$ là dữ liệu dạng time series đã được gán nhãn.
	\item $s$ là dữ liệu dạng time series đang được phân tích.
	\item $r_{i}$ và $s_{i}$ là các giá trị của dữ liệu tại thời điểm $i$ trong
	tập dữ liệu $r$ và $s$ có độ dài $N$.
\end{itemize}

Nếu $r$ có độ dài lớn hơn $s$, ta lấy giá trị nhỏ nhất trong các khoảng cách
$d(r_s, s)$, với $r_s$ là tập con (sub-series) của $r$. Với hàm khoảng cách ở
trên, ta có thể tính trọng lượng (weight) bằng cách:
\[ W(r,s)=e^{-\lambda \cdot d(r,s)} \]

Tham số $\lambda$ là một hệ số để điều chỉnh mức độ quan trọng giữa các time
series khác nhau hoặc giống nhau. Ví dụ, giá trị $\lambda$ lớn cho ra $W$ rất
nhỏ kể cả khi khoảng cách $r$ và $s$ là rất lớn. Khi đó, $\eta$ sẽ được tính
bằng tỉ lệ:
\[
\eta(s)=\frac{\sum_{r \in R+} W(r, s)}{\sum_{r \in R-} W(r, s)}
\]

với $R+$ là tập dữ liệu gồm nhiều time series được gán nhãn \textit{trend} và
$R-$ là dữ liệu được gán nhãn \textit{không trend}. Giá trị $\eta$ càng cao thì
time series $s$ càng \textit{giống trend} và ngược lại.

Khó khăn chính khi hiện thực mô hình này là việc gán nhãn cho các dữ liệu
\textit{đang trend} và \textit{không trend}. Bên cạnh đó, để mô hình hoạt động
trên mọi kích thước dữ liệu, việc thực hiện biến đổi (transformation) trên các
tập dữ liệu là cần thiết. Các biến đổi có thể là~\cite{nikolovTrendNoTrend2012}:

\begin{itemize}
	\item \textbf{Chuẩn hoá đường cơ sở (Baseline Normalization)}: Quan sát cho
		thấy khá nhiều chủ đề \textit{không trend} có tần suất Tweet rất cao,
		bên cạnh đó cũng có rất nhiều chủ đề \textit{đang trend} lại có tần
		suất Tweet rất thấp. Ví dụ, chủ đề ``city'' có tần suất cơ sở (baseline
		rate) rất cao vì nó là một âm tiết phổ biến trong tiếng Anh. Vì vậy nên
		ta có thể sử dụng Baseline Normalization để nhấn mạnh (emphasize) các
		điểm dữ liệu cao hơn đường cơ sở và giảm độ quan trọng của các điểm dữ
		liệu thấp hơn đường cơ sở.
	
	\item \textbf{Chuẩn hoá spike (Spike Normalization)}: Các chủ đề
		\textit{đang trend} và các chủ đề \textit{không trend} còn khác nhau ở
		độ lớn và tần suất của các điểm spike (điểm có giá trị lớn đột ngột).
		Vì vậy, Spike Normalization giúp nhấn mạnh các điểm spike và ngược lại.
	
	\item \textbf{Thang đo lôgarit (Logarithmic Scale)}: Ta thường không thể
		biết được cách thức mà một chủ đề được lan toả ra khắp cộng đồng, nhưng
		ta biết được đa số các quá trình phân nhánh (branching process) đều
		phát sinh theo cấp số mũ. Vì vậy, ta có thể sử dụng thang đo lôgarit để
		định lượng, qua đó có thể khai thác các đặc điểm này.

\end{itemize}

Mặc dù dữ liệu được dán nhãn sẽ quyết định kết quả của một time series, nhưng
ta vấn có thể điều chỉnh mô hình bằng cách thay đổi các tham số như $\lambda$,
độ dài time series $s$ và $r$, chọn các phương thức biến đổi để đạt được kết
quả tốt nhất, vì nhiều trong số chúng có ảnh hưởng trực tiếp tới các chỉ số như
true-positive và false-positive~\cite{hendricksonTrendDetectionSocial2015}.

\section{Mô hình phát hiện chủ đề}
Chủ đề là vấn đề cơ bản, là nội dung trọng tâm mà người nói hoặc người viết
muốn đề cập. Một văn bản (bài báo, Tweet, bài văn, v.v.) có thể chứa nhiều chủ
đề, nhưng chỉ có một hoặc một vài chủ đề có thể làm chủ đề chính để đại diện
cho văn bản đó.

Để tìm ra chủ đề, ta không thể đơn giản chỉ tìm từ phổ biến nhất trong văn bản.
Mỗi từ tuỳ theo ngữ cảnh có thể thuộc về nhiều chủ đề khác nhau. Ví dụ như âm
tiết ``đường'' có thể vừa thuộc chủ đề ``gia vị'' và vừa có thể thuộc chủ đề
``công trình''. Mô hình phát hiện chủ đề là mô hình túi từ (bag-of-words model)
dùng để khai thác mối tương quan giữa các từ và các chủ đề ngữ nghĩa tiềm ẩn
(latent sematic theme) giữa các văn
bản~\cite{madaniRealtimeTrendingTopics2015}.

Về cơ bản, mô hình chủ đề thực hiện ba công
việc~\cite{kapadiaTopicModelingPython2020}:

\begin{itemize}
	\item \textbf{Giảm chiều (Dimensionality Reduction)}: thay vì biểu diễn từ
		T ở không gian đặc trưng (feature space) \{Word\_i: count(Word\_i, T)
		for Word\_i in Vocabulary\}, ta có thể biểu diễn nó thành \{Topic\_i:
		Weight(Topic\_i, T) for Topic\_i in Topics\}.
	
	\item \textbf{Học không giám sát (Unsupervised Learning)}: có thể so sánh
		với gom cụm, khi số lượng cụm hay số lượng chủ đề là tham số cho đầu ra
		của mô hình. Khi ta thực hiện mô hình chủ đề, ta gọm cụm các từ thay vì
		gom cụm các văn bản.
	
	\item \textbf{Gán nhãn (Tagging)}: tìm cách biểu diễn các chủ đề mà các văn
		bản trong một tập văn bản thể hiện.
\end{itemize}

\subsection{Mô hình Latent Dirichlet Allocation (LDA)}
Mô hình LDA là mô hình tạo sinh xác suất (generative probabilistic model) cho
các dữ liệu rời rạc như các kho văn bản (text corpora). LDA là mô hình Bayes ba
lớp, với mỗi văn bản trong một tập văn bản được biểu diễn thông qua một tập các
chủ đề. Và mỗi chủ đề được biểu diễn bằng một tập các xác suất chủ
đề~\cite{bleiLatentDirichletAllocation2003}. Nói cách khác, một văn bản có thể
biểu diễn bằng nhiều chủ đề, và một chủ đề có thể được biểu diễn bằng nhiều từ
khác nhau.

Giả sử ta có 1000 văn bản, và mỗi văn bản có trung bình 300 từ. Để biết được thể
loại của mỗi văn bản, ta có thể nối mỗi văn bản với các từ xuất hiện trong văn
bản đó (Hình \ref{fig:simple_word_doc_model}). Khi đó, một tập các văn bản kết
nối cùng với một tập các từ thì chúng được gọi là cùng chủ
đề~\cite{ganegedaraIntuitiveGuideLatent2021}. Tuy nhiên, số kết nối ta phải tính
toán là quá lớn: $1000 \cdot 300 = 300000$.

\image[0.7]{img/lda/simple_word_doc_model.jpeg}{Mô hình kết nối văn bản - từ đơn
giản. Ảnh:
Medium~\cite{ganegedaraIntuitiveGuideLatent2021}}{fig:simple_word_doc_model}

Để giảm số kết nối, LDA thêm một lớp tiềm ẩn (latent), tức là các chủ đề ẩn chứa
trong các văn bản đều bị ẩn đi (unknown)(Hình \ref{fig:latent_word_doc_model}).
Tuy nhiên chúng vẫn được thể hiện bằng cách tạo sinh văn bản dựa trên các chủ đề
đó~\cite{tomarTopicModelingUsing2019}.

\image[0.7]{img/lda/latent_word_doc_model.jpeg}{Mô hình có lớp tiềm ẩn giúp giảm
số lượng kết nối (thread). Ảnh:
Medium~\cite{ganegedaraIntuitiveGuideLatent2021}}{fig:latent_word_doc_model}

\image[0.7]{img/lda/architecture.png}{Minh hoạ mô hình LDA. Các node thể hiện
cho các biến ngẫu nhiên, các cạnh thể hiện sự phụ thuộc giữa các biến ngẫu
nhiên. Node màu xám là biến ngẫu nhiên có thể quan sát được; node màu trắng là
các biến ngẫu nhiên ẩn (hidden). Các hình chữ nhật là ký hiệu tấm (plate
notation), thể hiện sự lặp lại của các biến ngẫu nhiên
~\cite{bleiTopicModels2009}.}{fig:lda_architecture}

Hình \ref{fig:lda_architecture} minh hoạ mô hình LDA. Quá trình tạo sinh từng
văn bản $w$ trong kho văn bản $D$ được thực hiện như
sau~\cite{bleiLatentDirichletAllocation2003}:

\begin{enumerate}
	\item Chọn $N \sim Poisson(\xi)$.
	\item Chọn $\theta \sim Dir(\alpha)$ với $Dir$ là phân phối Dirichlet.
	\item Với mỗi từ $w_n$ trong N từ:
	\begin{enumerate}
		\item Chọn $z_n \sim Multinomial(\theta)$.
		\item Chọn một chủ đề $z_n$ từ $p(w_n|z_n,\beta)$, là một xác suất đa
		thức có điều kiện phụ thuộc vào chủ đề $z_n$.
	\end{enumerate}
\end{enumerate}

LDA là mô hình học không giám sát (unsupervised) nên dữ liệu có thể không cần
được gán nhãn mà vẫn đưa được vào mô hình. Về cơ bản, LDA sẽ cố gắng sinh ra
các văn bản giống các văn bản đầu vào nhất có thể, từ đó mô hình học được các
tham số $\alpha$ và $\beta$.

% # TODO \subsection{Phát hiện trend twitter với LDA}

\section{Phương pháp khai thác ngữ nghĩa}
\subsection{Word embedding, sentence embedding}
Véc-tơ ngữ nghĩa (vectors semantics) là phương pháp cơ bản để biểu diễn nghĩa
của các từ trong xử lý ngôn ngữ tự nhiên. Ý tưởng của véc-tơ ngữ nghĩa là
chuyển một từ thành một điểm trong không gian ngữ nghĩa (semantic space). Điểm
mà một từ thể hiện sẽ được rút ra từ sự phân bổ các điểm (từ) xung quanh nó
(neighbor). Véc-tơ dùng để biểu diễn một từ gọi là word
embedding~\cite{jurafskySpeechLanguageProcessing}.

\image[0.7]{img/bert/word-embedding.png}{Minh hoạ các véc-tơ ngữ nghĩa của từ,
các từ gần nhau thường sẽ mang ý nghĩa/ý nghĩa sử dụng gần nhau. Ảnh:
Medium~\cite{bujokasCreatingWordEmbeddings2020}.}{fig:word_embedding}

Sentence embedding cũng giống như word embedding, nhưng nó dùng để biểu diễn sự
tương quan ngữ nghĩa giữa các câu. Một cách đơn giản để tạo sentence embedding
là lấy trung bình tất cả các word embedding của các từ trong câu
đó~\cite{lamGomCumVan2021}. Sentence embedding giúp ta khai thác được ngữ cảnh rộng hơn, trải dài qua nhiều câu mà các word embedding không thể làm được.

Sentence embedding có thể sử dụng để phân loại (ví dụ, câu đang xét là mang ý
nghĩa tích cực hay tiêu cực), phân tích ngữ nghĩa (ví dụ, phân tích sự liên
quan giữa hai câu bất kỳ) hoặc dùng trong các tác vụ ngôn ngữ (linguistic task,
ví dụ, xác định động từ là thì hiện tại hay thì quá khứ trong tiếng
Anh)~\cite{heidenreichPaperSummaryEvaluation2018}.

\subsection{Mô hình BERT}
BERT~\cite{devlinBERTPretrainingDeep2019} (Bidirectional Encoder
Representations from Transformers) là mô hình ngôn ngữ dựa trên kiến trúc
transformer~\cite{vaswaniAttentionAllYou2017}. BERT được các kỹ sư ở Google
phát triển và công bố vào năm 2018, đánh dấu bước đột phá trong các tác vụ xử
lý ngôn ngữ tự nhiên và các benchmark như GLUE (đánh giá mức độ hiểu ngôn ngữ),
SQuAD, v.v.

BERT là mô hình học sẵn (pre-trained model) được huấn luyện sẵn cho mục đích
chung (general purpose). BERT được thiết kế để huấn luyện trước (pre-train) các word embedding. Đầu ra của BERT sẽ được tinh chỉnh (fine-tune) với các
lớp huấn luyện bổ sung để phù hợp với các yêu cầu cụ thể của các tác vụ khác
nhau.

BERT là mô hình ngôn ngữ hai chiều (bidirectional), tức là một từ trong câu sẽ
được biểu diễn bởi các từ xung quanh nó. Khác với các mô hình một chiều
(unidirectional), một từ trong câu chỉ xét các từ ở phía trước nó. Ví dụ cho mô
hình một chiều là ELMo~\cite{petersDeepContextualizedWord2018}; ``Mặc dù ELMo có
kiến trúc dựa trên một mạng BiLSTM xem xét bối cảnh theo hai chiều từ trái sang
phải và từ phải sang trái nhưng những chiều này là độc lập nhau nên ta coi như
đó là biểu diễn một chiều.''~\cite{khanhBERTModel2020}.

BERT chỉ sử dụng phần encoder của kiến trúc transformer (sẽ được trình bày ở
phần kế tiếp). Khác với Recurrent Neural Network (RNN), cơ chế attention của
transformer truyền tất cả các từ trong câu vào mô hình và thực hiện đánh giá
(evaluate) chúng cùng một lúc, do đó transformer có thể coi là huấn luyện hai
chiều~\cite{khanhBERTModel2020}.

Đầu vào của BERT là một tập các từ được encode thành các véc-tơ embedding. Trước tiên, đầu vào sẽ được WordPiece phân tách thành các token. BERT sẽ dựa trên các token để tạo ra ba véc-tơ embedding:
\begin{itemize}
	\item Các token sẽ được chuyển thành các véc-tơ 768 chiều, trở thành Token
		Embedding.
	\item Segment Embedding là các véc-tơ để phân biệt các câu. Giữa các câu
		này sẽ được phân chia bởi một token [SEP]. Ví dụ, véc-tơ Segment
		Embedding của câu thứ nhất sẽ mang giá trị 0 ở tất cả các chiều, và
		véc-tơ Segment Embedding của câu thứ hai sẽ mang giá trị 1 ở tất cả các
		chiều.
	\item Positional Embedding sẽ cho biết token đang nằm ở vị trí nào trong
		câu và được tính bằng các hàm $sin$ và $cos$ ở các tần suất khác
		nhau~\cite{vaswaniAttentionAllYou2017}.
\end{itemize}

\image[1]{img/bert/input.png}{Minh hoạ đầu vào của
BERT~\cite{devlinBERTPretrainingDeep2019}}{fig:bert_input}

Đầu vào cho các lớp tiếp theo của BERT sẽ là tổng của ba véc-tơ embedding trên.
Chiều dài tối đa của đầu vào cho mô hình là 512 token, token đầu tiên luôn là
token [CLS] (token classification). Và giữa các câu được phân chia bởi token
[SEP]. BERT được huấn luyện trên tập dữ liệu
BooksCorpus~\cite{zhuAligningBooksMovies2015} (800 triệu từ) và Wikipedia tiếng
Anh (2500 triệu từ).

BERT sử dụng hai tác vụ để huấn luyện:

\begin{enumerate}
	\item \textbf{Masked ML (MLM)}:
		Vì mô hình BERT là mô hình hai chiều nên một từ trong câu có thể ``thấy
		được chính nó'', qua đó mô hình có thể gian lận để dự đoán từ một cách
		dễ dàng~\cite{devlinBERTPretrainingDeep2019}. Chính vì vậy, ta cần phải
		che dấu (mask) một số từ để huấn luyện BERT dự đoán các từ bị che đó.
		Qua nhiều lần thử nghiệm, \cite{devlinBERTPretrainingDeep2019} chọn che
		dấu 15\% số từ trong câu và chúng được chọn một cách ngẫu nhiên. Tuy
		nhiên, vì quá trình fine-tune không sử dụng Masked ML để huấn luyện,
		BERT chỉ thay thế 80\% trong số 15\% đó bằng token [MASK]. Trong 20\%
		còn lại, một nửa sẽ được thay thế bởi các token ngẫu nhiên và nửa còn
		lại sẽ được giữ nguyên token ban đầu.
	\item \textbf{Next Sentence Prediction (NSP)}: Đầu vào của BERT sẽ là hai
		câu bất kỳ trong tập dữ liệu được phân cách bởi token [SEP]. BERT chọn
		câu thứ hai là câu tiếp theo (trên tập dữ liệu) của câu thứ nhất với cơ
		hội 50\%, và 50\% còn lại câu thứ hai được lấy ngẫu nhiên từ tập dữ
		liệu. BERT sẽ được huấn luyện để dự đoán xem câu thứ hai có phải là câu
		tiếp theo của câu thứ nhất hay không, qua đó huấn luyện BERT hiểu được
		mối quan hệ giữa các câu với nhau.
\end{enumerate}

Quá trình tinh chỉnh (fine-tune) cho các tác vụ khác được dựa trên mô hình BERT
đã được huấn luyện sẵn (pre-trained model). Ta có thể thêm một hoặc các lớp
(layer) bổ sung mà không cần can thiệp hay chỉnh sửa mô hình BERT ban đầu.

\subsection{Encoder của kiến trúc transformer} Kiến trúc
transformer~\cite{vaswaniAttentionAllYou2017} bao gồm encoder và decoder
(hình \ref{fig:transformer}). Về cơ bản, encoder được sử dụng để chuyển đầu vào
thành các đặc trưng đơn giản (feature learning). Decoder sử dụng kết quả của
encoder để dự đoán nhãn của đầu ra. Mô hình BERT chỉ sử dụng phần encoder của
kiến trúc transformer.

\image[0.6]{img/bert/transformer.png}{Kiến trúc transformer với phần bên trái
là encoder, bên phải là
decoder~\cite{vaswaniAttentionAllYou2017}.}{fig:transformer}

Encoder gồm hai thành phần chính là Multi-Head Attention và mạng nơ-ron truyền
thẳng (Feed-Forward Neural Network, hay FNN). Mỗi thành phần được bao quanh bởi
một residual connection~\cite{heDeepResidualLearning2016} và layer
normalization~\cite{baLayerNormalization2016}: $LayerNorm(x + Sublayer(x))$ với
$Sublayer$ là Multi-Head Attention hoặc FFN.

Cơ chế self-attention là cơ chế giúp các từ điều chỉnh trọng số của các từ sao
cho sự ảnh hưởng của các từ xung quanh đối với bản thân nó là hợp lý nhất. Ví
dụ câu ``Dũng là một vận động viên nên anh ấy chơi bóng rổ rất hay'', thì từ
\textit{Dũng} sẽ nên có ảnh hưởng nhiều đến từ \textit{anh ấy} hơn các từ còn
lại. Self-attention được tính bằng query, key và value: $Q = W_q \cdot x$, $K =
W_k \cdot x$, $V = W_v \cdot x$ với $W_q, W_k, W_v$ là các ma trận mà BERT sẽ
học trong quá trình huấn luyện, và $x$ là véc-tơ embedding cho đầu vào. $Q, K,
V$ sau đó sẽ được cho qua lớp Scaled Dot-Product Attention để tính giá trị
attention (hình~\ref{fig:multi_attention}):
\[ \operatorname{Attention}(Q, K, V)=\operatorname{softmax}\left(\frac{Q
K^{T}}{\sqrt{d_{k}}}\right) V \]

với $d_k$ là chiều của query và key. Multi-Head Attention bao gồm nhiều lớp
self-attention, và được sinh ra bằng cách nối (concatenate) các lớp
self-attention đó lại với nhau (hình~\ref{fig:multi_attention}):
\[ MultiHead(Q,K,V) = Concat(head_1,...,head_h)W^O \]
\[ \text{với } head_i = Attention(QW_i^Q, KW_i^K, VW_i^V). \]

Với $W_O$, $W_i^Q$, $W_i^K$ và $W_i^V$ là các linear projection, cũng là các ma
trận tham số (parameter matrix) được học trong quá trình huấn luyện.

\image[1]{img/bert/multi-attention.png}{Minh hoạ Scaled Dot-Product Attention
và Multi-Head
Attention~\cite{vaswaniAttentionAllYou2017}.}{fig:multi_attention}

FFN là mạng nơ-ron bao gồm hai biến đổi tuyến tính sử dụng hàm kích hoạt ReLU
(ReLU activation function):
\[ FFN(x) = max(0,xW_1 + b_1)W_2 + b_2 \]

Kiến trúc của BERT bao gồm nhiều khối encoder chồng lên nhau. Nếu $L$ là số
khối encoder, $H$ là kích thước của véc-tơ embedding, $A$ là số self-attention
head trong lớp Multi-Head Attention, BERT đề nghị hai kích thước mô hình
chính~\cite{devlinBERTPretrainingDeep2019}:
\begin{itemize}
	\item \textbf{BERT\textsubscript{BASE}} (L=12, H=768, A=12, 110 triệu tham
		số).
	\item \textbf{BERT\textsubscript{LARGE}} (L=24, H=1024, A=16, 340 triệu
		tham số).
\end{itemize}

Ngoài ra còn các mô hình pre-train mở rộng của BERT như \textbf{BERT
Multilingual} (hỗ trợ nhiều ngôn ngữ) hay
\textbf{DistilBERT}~\cite{sanhDistilBERTDistilledVersion2020} (mô hình nhẹ và
nhanh hơn BERT) và RoBERTa~\cite{liuRoBERTaRobustlyOptimized2019} (mô hình khai
thác tối ưu phương thức huấn luyện của BERT), v.v.

\subsection{PhoBERT, mô hình BERT cho tiếng Việt}
PhoBERT~\cite{nguyenPhoBERTPretrainedLanguage2020} (hay Phở BERT) được cho là
đạt hiệu quả cao hơn so với mô hình BERT Mutilingual trong các tác vụ xử lý
ngôn ngữ tiếng Việt. PhoBERT được thiết kế dựa trên
RoBERTa~\cite{liuRoBERTaRobustlyOptimized2019}. RoBERTa là nghiên cứu mô phỏng
(replication study) của mô hình BERT, các tham số của RoBERTa và các kích thước
tập huấn luyện được đánh giá (evaluate) một cách kỹ lưỡng hơn. Các nhà nghiên
cứu phía sau RoBERTa cảm thấy BERT chưa được huấn luyện để sử dụng hết khả năng
của nó (undertrained)~\cite{liuRoBERTaRobustlyOptimized2019}, và đã đề nghị một
công thức mới để huấn luyện BERT. Các thay đổi bao
gồm~\cite{liuRoBERTaRobustlyOptimized2019}: \textbf{(1)} huấn luyện mô hình lâu
hơn, với kích thước batch lớn hơn và dữ liệu nhiều hơn; \textbf{(2)} không sử
dụng Next Sentence Prediction để huấn luyện; \textbf{(3)} huấn luyện trên các
câu dài hơn; và \textbf{(4)} liên tục thay đổi cơ chế ẩn dấu (mask) token trong
quá trình huấn luyện.

Đầu vào của PhoBERT cần qua bước xử lý bởi
VnCoreNLP~\cite{vuVnCoreNLPVietnameseNatural2018}. VnCoreNLP là bộ công cụ xử
lý ngôn ngữ tự nhiên, hỗ trợ các tác vụ như phân đoạn từ (word segmentation),
gán nhãn từ loại (part-of-speech tagging, hay POS), nhận dạng thực thể có tên
(named entity recognition, hay NER) và phân tích quan hệ phụ thuộc (dependency
parsing). PhoBERT chỉ yêu cầu sử dụng tác vụ phân đoạn từ từ VnCoreNLP. Phân
đoạn từ là tác vụ quan trọng tại vì tiếng Việt là ngôn ngữ đơn âm, mà các từ có
thể vừa là từ đơn vừa có thể là từ ghép. Cho nên các thành phần của từ ghép nếu
đứng riêng sẽ không có nghĩa hoặc có nghĩa khác so với từ ghép bao gồm nó.
