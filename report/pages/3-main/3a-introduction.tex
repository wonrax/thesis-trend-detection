\section{Giới thiệu về đề tài}
\label{sec:introduction}

\subsection{Đặt vấn đề}
Với sự bùng nổ của dữ liệu điện tử, đặc biệt là dữ liệu dạng văn bản trong thời
đại internet, chúng đem lại nhiều cơ hội và thách thức cho việc nghiên cứu và
phân tích dữ liệu. Liệu con người có bị chìm trong lượng thông tin khổng lồ như
vậy không? Liệu chúng ta có cách nào đó sử dụng sức mạnh tính toán của máy
tính, tự động hoá công việc phân tích các dữ liệu để đưa ra các thông tin cần
thiết và hữu ích?

Bên cạnh đó, sự phát triển của mạng xã hội trong thập kỷ 21 đã dần hình thành
nên các hành vi tâm lý của người dùng. Đặc biệt trong số đó là hội chứng sợ bỏ
lỡ (fear of missing out, hay FOMO), là cảm giác bứt rứt hoặc ám ảnh rằng bản
thân đang bỏ lỡ những việc mà những người cùng trang lứa có thể làm được hoặc
làm tốt hơn~\cite{jwtintelligenceFearMissingOut2015}. Chính vì các hành vi tâm
lý này, các mạng xã hội, cụ thể là các nhà khoa học dữ liệu đã tận dụng dữ liệu
để giúp người dùng có thể tìm kiếm thông tin một cách nhanh chóng và hiệu quả.
Đơn cử là mạng xã hội Twitter với tính năng phát hiện Xu hướng (Trending now),
hay Đang xảy ra (What's Happening) giúp họ trở thành mạng xã hội phổ biến để
cập nhật tin tức từ chính trị, xã hội đến đời sống cá nhân của những người nổi
tiếng.

Để phát hiện sự thay đổi của các đề tài đang nổi bật trong đại chúng, về cơ bản
ta cần phát hiện được chủ đề của các văn bản và tìm cách biến chúng thành các
thể hiện con người có thể hiểu được. Tiếp theo ta cần phân tích mức độ phổ biến
của các chủ đề, hay phạm vi mà chủ đề đó đang được phát triển để quyết định
xem các chủ đề đó có đáng được quan tâm hay không.

Cho đến hiện tại, đã xuất hiện nhiều mô hình phát hiện chủ đề và phân tích xu
hướng và có thể chia thành mô hình xác suất và mô hình học máy. Mô hình xác
suất về cơ bản là đếm các từ khoá và sử dụng phân phối xác suất để xác định bất
thường trong sự thay đổi về mức độ phổ biến của các từ khoá đó. Về mô hình học
máy ta có thể chia ra làm hai loại: Mô hình học máy sử dụng tập dữ liệu huấn
luyện có chứa thông tin về mức độ thay đổi được gán nhãn là trending hoặc không
trending; Hoặc mô hình phân tích ngữ nghĩa của văn bản dùng để phát hiện chủ đề
của một tập dữ liệu ta quan tâm.

Cần vận dụng đúng mô hình để phát hiện chủ đề NHANH CHÓNG ,...., không Cần
tập dữ liệu gán nhãn,...
\# TODO viet tiep

\subsection{Mục tiêu}
\begin{itemize}
	\item Tìm hiểu các cách thức phát hiện xu hướng dựa trên các mô hình xác
		suất hoặc mô hình hướng dữ liệu, đơn cử là mô hình phân phối Poisson.

	\item Tìm hiểu các mô hình phát hiện chủ đề (Latent Dirichlet Allocation
		hay LDA,...).

	\item Tìm hiểu các mô hình phát hiện chủ đề dựa trên ngữ nghĩa (BERT,
		PhoBERT,...)
	
	\item Thu thập dữ liệu để phục vụ mục đích nghiên cứu.
	
	\item Thử nghiệm các mô hình với các tập dữ liệu đã thu thập.
\end{itemize}
