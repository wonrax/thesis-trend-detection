\documentclass[11pt, a4paper]{article}
\usepackage[a4paper, total={6in, 9in}]{geometry}
\usepackage{iftex}
% package for including graphics with figure-environment
\usepackage{graphicx}
\usepackage[unicode]{hyperref}  

% For subfigure
\usepackage{subcaption}
% colors for hyperlinks
% colored borders (false) colored text (true)
\hypersetup{colorlinks=true, citecolor=black, filecolor=black, linkcolor=black, urlcolor=black}

% package for bibliography
\usepackage[numbers]{natbib}

% package for header
% \usepackage[automark,headsepline]{scrlayer-scrpage}
% \pagestyle{scrheadings}
% \ihead[]{Name of students}
% \ohead[]{\today}
% \cfoot[]{\pagemark} 

% Remove hyphenation warning.
\usepackage[vietnamese=nohyphenation]{hyphsubst}

\ifPDFTeX
	\usepackage[utf8]{inputenc}
	\usepackage[vietnamese]{babel}
\else
	% Other engines (e.g. XeTeX or LuaTeX)

	\usepackage[vietnamese]{babel}
	\usepackage{fontspec}
	\usepackage{unicode-math}
	\setmonofont{Inconsolatazi4}
	\setmainfont{texgyrepagella-regular}[
		Path			= ./qpl2_501otf/,
		Extension		= .otf,
		BoldFont		= texgyrepagella-bold,
		ItalicFont		= texgyrepagella-italic,
		BoldItalicFont	= texgyrepagella-bolditalic
		]
	\setmathfont{texgyrepagella-math}[
		Path			= ./qpl2_501otf/,
		Extension		= .otf
		]
\fi

% Table of content verticle spacing
\usepackage{tocloft}
\renewcommand\cftparskip{0.2cm}

\newcommand{\image}[3][1]{
\begin{figure}[ht]
	\includegraphics[width=#1\textwidth,keepaspectratio]{#2}
	\centering
	\caption{#3}
\end{figure}
} 

\newcommand{\doubleimage}[7][0.4]{ 
\begin{figure}[ht]
	\centering
	\begin{subfigure}{.5\textwidth}
		\centering
		\includegraphics[width=#1\linewidth]{#2}
		\caption{#3}
	\end{subfigure}%
	\begin{subfigure}{.5\textwidth}
		\centering
		\includegraphics[width=#1\linewidth]{#4}
		\caption{#5}
	\end{subfigure}
	\caption{#6}
\end{figure} 
}

\setlength{\parindent}{2em}

\setlength{\parskip}{0.7em}

% Line spacing
\renewcommand{\baselinestretch}{1.3}

\begin{document}

\title{
	{\parskip=0.4em\normalsize
			Đại học Quốc gia Thành phố Hồ Chí Minh\\
			Trường Đại học Bách Khoa\\
			Khoa Khoa học \& Kỹ thuật Máy tính\\
		}
	\vspace{0.5cm}
	\begin{figure}[!ht]
		\centering
		\includegraphics[width=0.26\textwidth]{img/logo/LogoBKChinhThuc.png}
	\end{figure}
	\vspace{0.1cm}
	\Large {Báo cáo Đề cương luận văn} \\
	\vspace{0.3cm}

	\Huge {\renewcommand{\baselinestretch}{0.1} Phát hiện chủ đề và phân tích\\[-0.2em]xu hướng}

	\vspace{0.3cm}
}
\author{}

% if you are the only author, you might use the following
% \author{Name of student}	

% Insert here your name and correct mail address

% name of the course and module
\date{
	\textbf{\large GVHD}\\[0.3em]
	Lê Thanh Vân\\[2em]
	\textbf{\large SVTH}\\[0.3em]
	Hà Huy Long Hải - 1812064\\
	\vspace{4cm}
	\normalsize{\today}
}

\maketitle
\thispagestyle{empty}

\vspace{2cm}

\begin{abstract}
	Lorem ipsum dolor sit amet, consectetur adipiscing elit. Morbi in dictum mauris, vel maximus enim. Nulla ultrices dui id mi luctus suscipit. Pellentesque nec dignissim arcu, id commodo eros. Nullam non rutrum risus. Donec viverra finibus velit. In convallis, massa eget suscipit feugiat, urna magna cursus ligula, eu pretium neque risus eget lectus. Duis sed dui quis nibh ullamcorper molestie ac in felis.
\end{abstract}

\newpage
\tableofcontents
\newpage

\section{Giới thiệu về đề tài}
\label{sec:introduction}

\subsection{Đặt vấn đề}
Với sự bùng nổ của dữ liệu điện tử, đặc biệt là dữ liệu dạng văn bản trong thời đại internet, chúng
đem lại nhiều cơ hội và thách thức cho việc nghiên cứu và phân tích dữ liệu. Liệu con người có bị
chìm trong lượng thông tin khổng lồ như vậy không? Liệu chúng ta có cách nào đó sử dụng sức mạnh
tính toán của máy tính, tự động hoá công việc phân tích các dữ liệu để đưa ra các thông tin cần
thiết và hữu ích?

Bên cạnh đó, sự phát triển của mạng xã hội trong thập kỷ 21 đã dần hình thành nên các hành vi tâm
lý của người dùng. Đặc biệt trong số đó là hội chứng sợ bỏ lỡ (fear of missing out, hay FOMO), theo
\cite{jwtintelligence_fear_2015}, là cảm giác bứt rứt hoặc ám ảnh rằng bản thân đang bỏ lỡ những
việc mà những người cùng trang lứa có thể làm được hoặc làm tốt hơn. Chính vì các hành vi tâm lý
này, các mạng xã hội, cụ thể là các nhà khoa học dữ liệu đã tận dụng dữ liệu để giúp người dùng có
thể tìm kiếm thông tin một cách nhanh chóng và hiệu quả hơn. Đơn cử là mạng xã hội Twitter với
tính năng phát hiện Xu hướng (Trending now), hay Đang xảy ra (What's Happening) giúp họ trở thành
mạng xã hội phổ biến để cập nhật tin tức từ chính trị, xã hội đến đời sống cá nhân của những người
nổi tiếng.

Để phát hiện sự thay đổi của các đề tài đang nổi bật trong đại chúng, về cơ bản ta phải phát hiện
được chủ đề của các văn bản, tức là phân tích xem họ đang viết về cái gì. Tiếp theo ta cần phân
tích mức độ phổ biến của các chủ đề đó, hay phạm vi mà chủ đề đó đang được phát triển để quyết định
xem các chủ đề đó có đáng được quan tâm hay không.

\subsection{Mục tiêu}
\begin{itemize}
	\item Tìm hiểu các cách thức phát hiện xu hướng dựa trên các mô hình xác suất hoặc mô hình
	hướng dữ liệu, đơn cử là mô hình phân phối Poisson.

	\item Tìm hiểu các mô hình phát hiện chủ đề (Latent Dirichlet Allocation hay LDA,...).

	\item Tìm hiểu các mô hình phát hiện chủ đề dựa trên ngữ nghĩa (BERT, PhoBERT,...)
\end{itemize}

\section{Các mô hình}
\label{sec:models}

\subsection{Phát hiện xu hướng}

\subsection{Phát hiện chủ đề}

\subsection{Phân tích ngữ nghĩa}

\subsection{VnCoreNLP}

\section{Thử nghiệm}
\label{sec:experiments}

\clearpage

\bibliographystyle{vietnumeric}
\bibliography{references} % expects file "references.bib"
\addcontentsline{toc}{section}{Tham khảo}


\end{document}
